\documentclass{article}
\usepackage{graphicx} % Graficas.
\usepackage[utf8]{inputenc} % Uso de UTF.
\usepackage[hidelinks]{hyperref} % Indice con hipervinculos.
\usepackage{caption}
\usepackage{listings} % Copeo de codigo.
\usepackage{fullpage} % Uso de pagina completa.
\usepackage{color} % Color para codigo.

\renewcommand*\contentsname{Índice}

\begin{document}

\begin{figure}[!htb]
\minipage{0.32\textwidth}
	\includegraphics[width=\linewidth]{"img/LogoSEP".png}
\endminipage\hfill
\minipage{0.32\textwidth}
	\includegraphics[width=\linewidth]{"img/LogoTNM".PNG}
\endminipage\hfill
\minipage{0.32\textwidth}
	\includegraphics[width=\linewidth]{"img/LogoITT".png}
\endminipage\hfill
\end{figure}

\begingroup
\LARGE
\begin{verbatim}
Subdirección Académica
Departamento de Sistemas y Computación
Ingeniería en Sistemas Computacionales
Semestre: Enero - Junio 2017
Materia: Sistemas Programables (3SC8A)

Nombre del tema:
Documentacion proyecto

Nombre de los integrantes:
Salcedo Morales José Manuel (13211419)
Espinoza Covarrubias Silverio Alejandro (13211465)
Alvarez Corral Miguel Angel (13211384)


Nombre del catedrático:
Ingeniero Luís Alberto Mitre Padilla

\end{verbatim}
\endgroup

\newpage
\tableofcontents

\newpage
\section{Introducción}

\section{Componentes utilizados}
\begin{itemize}
        \item Arduino
        \item Cables Jumper
        \item Fuente de alimentacion para arduino
\end{itemize}

\newpage
\section{Marco Teórico}
\begin{itemize}
	\item Arduino: Arduino se refiere a una plataforma o placa de electrónica de código abierto y al software utilizado para programarlo. Arduino está diseñado para hacer la electrónica más accesible a los artistas, diseñadores, aficionados y a cualquiera interesado en la creación de objetos interactivos o entornos. Un tablero de Arduino se puede comprar pre-ensamblado o, porque el diseño de hardware es de código abierto, construido a mano. De cualquier manera, los usuarios pueden adaptar las tablas a sus necesidades, así como actualizar y distribuir sus propias versiones.
\end{itemize}

\newpage
\section{Desarrollo}

\subsection{Imagenes}

\newpage
\section{Conclusión}

\renewcommand\refname{Referencias}
\begin{thebibliography}{1}
	\bibitem{Arduino} What is Arduino? - Definition from Techopedia. (n.d.). Retrieved March 26, 2017, from https://www.techopedia.com/definition/27874/arduino	
\end{thebibliography}

\end{document}
